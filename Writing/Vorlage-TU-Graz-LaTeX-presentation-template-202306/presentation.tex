\documentclass[aspectratio=169]{beamer}
%% Choose aspect ratio and other standard options:
% [aspectratio=169] % 16:9 (default)
% [aspectratio=43]  % 4:3 

\usetheme[minimal]{tugraz2018}
%% Choose main theme variant:
% [standard]        % standard (default)
% [institute]       % with institute's graphical acronym on the left
% [minimal]         % with reduced visuals

%% Choose your font style:
%                   % Helvetica (default for Corporate Design)
% [webfont]         % Source Sans Pro (as used on tugraz.at)
% [nofont]          % no font loaded - Computer Modern Sans

%% For more options, see README.pdf

\usepackage[utf8]{inputenc}
\usepackage[english]{babel}
\usepackage{todonotes}  % Add todonotes package
%% Choose your main language:
% [ngerman]   % German
% [english]   % English


%% Add your own packages, macros, etc.
% ...


%% Enter presentation metadata
\title[Skin Image Classification]{Skin Image Classification Pipeline:\\Using SAM2 Encoder Features}
\author{Taisiya Parkhomenko (01650051)}
\date{Bachelor Thesis}
\institute{Institute of Visual Computing (IVC)}
\instituteurl{www.ivc.tugraz.at}

%% Logos
\institutelogo{beamerthemetugraz/institute/kurz}  % IVC logo
% \additionallogo{figures/logo}  % additional logo if needed
% \logobar{Supported by: ...}  % sponsors (titlepage; optional)


\begin{document}

\begin{frame}[plain]
  \maketitle
\end{frame}


\begin{frame}{Outline}
  \tableofcontents
\end{frame}


\section{Problem Definition \& Motivation}

\begin{frame}{Problem Definition \& Motivation}
  \begin{itemize}
    \item Skin lesion classification is key for early melanoma detection.
    \item Datasets are imbalanced: some classes have 20k+, others <500.
    \item OOD risk: flowers, textures, artifacts misclassified as lesions.
    \item Goal: Build a robust multi-stage pipeline that handles imbalance and rejects OOD.
  \end{itemize}
  \vspace{0.5em}
  \todo[inline]{Add dataset histogram (class imbalance)}
  \todo[inline]{Add motivating example: lesion vs flower misclassification}
\end{frame}


\section{Pipeline Design}

\begin{frame}{Pipeline Design}
  \begin{itemize}
    \item Frozen SAM encoder extracts 256-dim features.
    \item Three downstream MLPs:
      \begin{itemize}
        \item MLP1: Skin vs Not-Skin
        \item MLP2: 8 Lesion classes
        \item MLP3: Benign vs Malignant
      \end{itemize}
    \item Compared Parallel vs Multihead setups.
      \begin{itemize}
        \item Parallel = independent MLPs
        \item Multihead = shared trunk + 3 heads
      \end{itemize}
  \end{itemize}
  \todo[inline]{Diagram: SAM encoder \textrightarrow{} trunk \textrightarrow{} 3 heads vs 3 separate MLPs}
\end{frame}


\section{Data Preparation \& Balancing}

\begin{frame}{Data Preparation \& Balancing}
  \begin{itemize}
    \item Exp1--5: Original ISIC splits (imbalanced).
    \item Exp6: Smart balancing strategy:
      \begin{itemize}
        \item Cap NV/UNKNOWN at 5k, BCC/MEL at 3k.
        \item Augment rare classes to 1k.
      \end{itemize}
    \item Augmentations: flips, rotations, jitter.
  \end{itemize}
  \vspace{0.5em}
  \todo[inline]{Table: per-class counts before vs after balancing}
  \todo[inline]{Images: example augmented lesion}
\end{frame}


\section{Loss Functions \& Training}

\begin{frame}{Loss Functions \& Training Strategies}
  \begin{itemize}
    \item Weighted Cross-Entropy: Baseline.
    \item Focal Loss: From literature, but unstable.
    \item LMF (LDAM + Focal Mix): Improves minority class margins.
    \item Balanced MixUp: Soft-label augmentation.
  \end{itemize}
  \vspace{0.5em}
  \todo[inline]{Equation slide: CE, Focal, LMF side by side}
  \todo[inline]{Bar chart: lesion macro-F1 by loss type}
\end{frame}


\section{Results \& Key Insights}

\begin{frame}{Results \& Key Insights}
  \begin{itemize}
    \item Parallel > Multihead for lesion classification.
    \item Best lesion: Parallel + LMF (Exp1) $\sim$65\% acc, 0.60 macro-F1.
    \item Skin detection near-perfect ($\sim$99\%).
    \item Benign/Malignant stable ($\sim$75--79\% acc).
  \end{itemize}
  \todo[inline]{Table: Parallel vs Multihead results}
  \todo[inline]{Confusion matrix for lesion task highlighting minority errors}
\end{frame}


\section{OOD Detection -- ODIN}

\begin{frame}{OOD Detection with ODIN}
  \begin{itemize}
    \item Manual test: flower misclassified as melanoma.
    \item Implemented ODIN (temperature scaling + input perturbation).
    \item Evaluated with Places365 dataset as OOD.
    \item Results: High AUROC/AUPR $\Rightarrow$ better safety.
  \end{itemize}
  \todo[inline]{ROC/AUPR curves comparing ODIN vs baseline}
  \todo[inline]{Diagram: ODIN workflow on top of MLP1}
\end{frame}


\section{Extra Experiment: Cropped Lesion Masks}

\begin{frame}{Extra Experiment: Cropped Lesion Masks}
  \begin{itemize}
    \item From S2R2: Segment to Recognize (Janou\v{s}kov\'a et al.).
    \item Motivation: Remove background noise (skin tone, artifacts, rulers).
    \item Plan: Train with SAM-cropped lesion patches.
    \item Status: Not done yet $\rightarrow$ future extension.
  \end{itemize}
  \todo[inline]{Image: original vs cropped lesion mask overlay}
\end{frame}


\section{Best Model (So Far)}

\begin{frame}{Best Model (So Far)}
  \begin{itemize}
    \item Parallel + LMF (Exp1) is best so far:
      \begin{itemize}
        \item Skin: 0.999 / 0.999
        \item Lesion: 0.645 / 0.599
        \item Benign/Malignant: 0.795 / 0.790
      \end{itemize}
    \item Strong balance across tasks compared to others.
  \end{itemize}
  \todo[inline]{Table: summary of best model metrics}
\end{frame}


\section{Future Directions}

\begin{frame}{Future Directions}
  \begin{itemize}
    \item Use MedSAM / UniSeg for medical segmentation features.
    \item Test cropped vs full images.
    \item Ensemble models (Parallel + Multihead).
    \item Try higher resolutions (512 vs 1024).
    \item Semi-supervised learning with unlabeled dermoscopy.
    \item Stronger OOD rejection (energy-based, Mahalanobis).
  \end{itemize}
  \todo[inline]{Bullet slide: future research directions with icons}
\end{frame}

\begin{frame}{Thank You}
  \centering
  \Large Thank you for your attention!
  
  \vspace{1cm}
  \normalsize
  \begin{itemize}
    \item Supervisor: Dr. Marc Masana Castrillo
    \item Institute of Visual Computing (IVC)
    \item Technical University of Graz
  \end{itemize}
\end{frame}

\end{document}
