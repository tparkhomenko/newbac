%----------------------------------------------------------------------
% Template for TECHNICAL REPORTS at Inst. of Visual Computing (IVC), Graz University of Technology, Austria:
% style file 'techrep_ivc.sty'
%
% author:  Pierre Elbischger
% email:   pierre.elibschger@icg.tu-graz.ac.at
%
% created: 13.11.2003
% revised: 29.04.2017 (Peter M. Roth, pmroth@icg.tugraz.at)
% revised: 15.09.2017 (Horst Possegger, possegger@icg.tugraz.at)
% modified: 08.07.2024 (Peter Mohr-Ziak, pmohr@tugraz.at)
% modified: 19.12.2024 (Peter Mohr-Ziak, pmohr@tugraz.at) IVC Version
%----------------------------------------------------------------------
% The template contains a number of LaTeX commands of the form :
%
% \command{xyz}
%
% In order to complete this template, fill in the blank fields between
% the curly braces or replace already filled in fields with the
% requested information.
%
% e.g. in order to add a title, replace
% \title{} with \title{Evidence of Solitons in Tedium Diboride}
%
% The \author and \address commands can take an optional label
% in square brackets of the form :
%
% \command[label]{}
%
% The text of the abstract should be inserted between the two commands
% \begin{abstract} and \end{abstract}.
%
% Please leave all commands in place even if you don't fill them in.


%----------------------------------------------------------------------
% Do not alter the following two lines

\documentclass[11pt, a4paper]{article}
\usepackage{techrep_ivc}

 
%----------------------------------------------------------------------
% Include any additional packages here!
% package 'graphicx' is automatically included depending on the
%   used compiler (latex, pdflatex), don't include it!!!
\usepackage{pdfpages}

\begin{document}
%----------------------------------------------------------------------
% Add the following information, which will be shown on the cover sheet

% Uncomment matching document type
% \type{Technical Report}
% \type{Seminar/Project Computer Vision}
% \type{Seminar/Project Computer Graphics}
% \type{Master Project}
\type{Bachelor Thesis}
%\type{Seminar Paper}

% Number of the technical report (if type == Technical Report)
\reportnr{xxx}

% Name of advisor (if type != Technical Report)
\advisor{Marc Masana}

% Title
\title{Skin Lesion Classification with ML Pipeline}

% Subtitle
\subtitle{}

% City where the report was created
\repcity{Graz}

% Date of creation
\repdate{\today}

% Keywords that appear below the abstract (if required)
\keywords{Skin Lesion, Classification, Computer Vision, Machine Learning}

% List of authors: List each author using a separate \author{} command. If there 
% is more than one author address, add a label to each author of the form 
% \author[label]{name}. This label should be identical to the corresponding 
% label provided with the \address command. It is not possible to link an author 
% to more than one address!
%
\author[IVC]{Taisiya Parkhomenko}
%\author[IVC]{Your Name 2}

% List of addresses:  If there is more than one address, list each using a 
% separate \address command using a label to link it to the respective author
% as described above.
%
\newcommand{\TUGn}{Graz University of Technology}
\address[IVC]{Institute of Visual Computing \\ \TUGn, Austria}


% Contact author: If \contact is not defined (uncommented) or empty, the contact  
% information on the title page is suppressed.
%
\contact{Taisiya Parkhomenko}
\contactemail{taisiya.parkhomenko@student.tugraz.at}

%----------------------------------------------------------------------
% Abstract:
\begin{abstract}

We propose a multi-step pipeline for automatic skin lesion classification. First, SAM2 image encoder features are extracted and used for skin vs. non-skin classification. For skin images, lesion type (e.g., keratosis, carcinoma) and benign vs. malignant are predicted. Segmentation masks provide interpretability and localization support.

\end{abstract}
%----------------------------------------------------------------------


%----------------------------------------------------------------------
% Main document 

% pipeline figure
\begin{figure}[t]
    \centering
    \includegraphics[width=0.85\linewidth]{figures/pipeline.png}
    \caption{Proposed skin lesion classification pipeline.}
\label{fig:pipeline}
\end{figure}

\section{Introduction}

Skin lesion classification is an active research topic in medical image analysis. We outline the motivation and the overall objective of our pipeline.

\section{Related Work}
Brief overview of prior work in lesion classification and segmentation.

\section{Method}
We describe the pipeline stages: feature extraction using SAM2, initial skin detection, lesion categorization, and malignancy prediction, with optional segmentation for interpretability.

\section{Results}
Present results (to be added) and discuss performance.

\section{Conclusion}
Summarize findings and future work.

% Attachments (from previous application materials)
\clearpage
\appendix
\section*{Appendix}
\addcontentsline{toc}{section}{Appendix}
\includepdf[pages=-]{Summer-Bachelor-Program-2025-Taisiya-Parkhomenko.pdf}
\includepdf[pages=-]{Transcript_de_UF 033 253_01650051.pdf}

%----------------------------------------------------------------------
% Bibliography/References
\bibliographystyle{plain}
\bibliography{references}

\end{document}
%----------------------------------------------------------------------

