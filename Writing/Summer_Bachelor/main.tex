\documentclass[10pt,a4paper]{article}
\usepackage[utf8]{inputenc} 
\usepackage[T1]{fontenc}
\usepackage[english]{babel}
\usepackage{lmodern}
\usepackage{graphicx}
\usepackage{xcolor}
\usepackage{geometry}
\usepackage{hyperref}
\usepackage{pdfpages}
\geometry{left=2.5cm, right=2.5cm, top=4cm, bottom=2.5cm}
\usepackage{fancyhdr}
\usepackage{sectsty}
\usepackage{enumitem}

% Hyperref setup for clickable links
\hypersetup{
    colorlinks=true,
    linkcolor=darkblue,
    filecolor=darkblue,
    urlcolor=darkblue,
}

% Define dark blue color
\definecolor{darkblue}{RGB}{0,0,139}
\definecolor{black}{RGB}{0,0,0}
% Header styling with dark blue
\sectionfont{\color{darkblue}\bfseries\Large}
\subsectionfont{\color{darkblue}\bfseries}
\pagestyle{fancy}
\fancyhf{}
% Left header: Application information
\lhead{\textcolor{black}{Application for:\\
Summer Bachelor Project at the IVC Institute\\
Graz University of Technology\\
May 7, 2025}}
% Right header: Name and contact information
\rhead{\textcolor{darkblue}{Taisiya Parkhomenko\\
\href{tel:+43-664-156-0-160}{+43 664 156 0 160}\\
\href{mailto:taisiyparkhomenko@gmail.com}{taisiyparkhomenko@gmail.com}\\
\href{https://www.linkedin.com/in/taisiya-parkhomenko-937336167/}{LinkedIn}}}
\renewcommand{\headrulewidth}{0pt}
\setlength{\headheight}{46.27916pt}
\addtolength{\topmargin}{-34.27916pt}

\begin{document}
\fancyfoot[C]{\thepage}
\pagestyle{fancy}

\vspace{4em}
\section*{\textcolor{darkblue}{\textbf{Motivation Letter}}}
\vspace{1em}

\noindent Dear selection committee, dear Prof. Plopski,

\vspace{1.5em}

I would love to participate in the summer Bachelor program at the IVC institute. I'm currently finishing my Bachelor's degree in \textbf{Biomedical Engineering} at TU Graz and have already started working on my thesis under the supervision of \textbf{Marc Masana}.

The project is still in its early stages, but I've soft-started the implementation. My current idea focuses on \textbf{skin lesion classification} using \textbf{machine learning}. The pipeline would be structured with multi-step decisions: First, I use a \textit{frozen encoder} to extract features from images, then a small neural network classifies them as \textbf{skin} or \textbf{non-skin}. If the image is classified as skin, I apply two additional classifiers – one for \textbf{lesion type} and one for benign vs. malignant prediction.

I've also been experimenting with \textbf{segmentation masks} to support the visual analysis and better localize relevant features. The plan is to integrate these steps into a pipeline that could be adapted or extended later on.

A few years ago, I worked at a company where we used ML in image processing – I really enjoyed that experience, comparing different models and exploring what works best. Since then, I've been programming regularly and looking forward to applying my skills in a \textit{medical context}.

I'm very motivated to better my knowledge in \textbf{Computer Vision} area and would be excited to work on it during this summer program.

\vspace{1.5em}

\noindent Thank you and best regards,\\
Taisiya Parkhomenko

\vspace{2em}
\hrule
\vspace{2em}

\section*{\textcolor{darkblue}{Commitment of Supervision}}
\vspace{1em}

\textit{Commitment of supervision of the corresponding advisor will be input here.}

\vspace{2em}
\hrule
\vspace{2em}

\newpage
\section*{\textcolor{darkblue}{Thesis Idea – Skin Lesion Classification with Multi-Stage ML Pipeline}}

I am currently working on my \textbf{Bachelor's thesis} under the supervision of \textbf{Marc Masana}. The goal is to build a \textbf{multi-stage pipeline} for automatic \textbf{skin lesion classification}.

\subsection*{\textbf{Pipeline Structure}}
\begin{enumerate}
    \item Use a \textbf{SAM2 encoder} to extract feature embeddings from input \textbf{skin images}.
    \item Classify images as \textbf{skin} vs. \textbf{non-skin} using a lightweight \textbf{MLP (MLP1)}.
    \item If classified as skin:
    \begin{itemize}
        \item Predict \textbf{lesion type} (e.g., keratosis, carcinoma) using \textbf{MLP2}.
        \item Predict \textbf{malignancy} (benign vs. malignant) using \textbf{MLP3}.
    \end{itemize}
    \item Apply \textbf{segmentation masks} to support visual interpretability and localize relevant regions.
\end{enumerate}

\vspace{2em}
\begin{figure}[htb!]
    \centering
    \includegraphics[width=0.7\textwidth]{pipeline.png}
    \caption{\textbf{System Architecture Diagram}}
\end{figure}


\includepdf[pages=-]{Transcript_de_UF 033 253_01650051.pdf}

\end{document}