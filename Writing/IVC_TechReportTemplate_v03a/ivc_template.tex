%----------------------------------------------------------------------
% Template for TECHNICAL REPORTS at Inst. of Visual Computing (IVC), Graz University of Technology, Austria:
% style file 'techrep_ivc.sty'
%
% author:  Pierre Elbischger
% email:   pierre.elibschger@icg.tu-graz.ac.at
%
% created: 13.11.2003
% revised: 29.04.2017 (Peter M. Roth, pmroth@icg.tugraz.at)
% revised: 15.09.2017 (Horst Possegger, possegger@icg.tugraz.at)
% modified: 08.07.2024 (Peter Mohr-Ziak, pmohr@tugraz.at)
% modified: 19.12.2024 (Peter Mohr-Ziak, pmohr@tugraz.at) IVC Version
%----------------------------------------------------------------------
% The template contains a number of LaTeX commands of the form :
%
% \command{xyz}
%
% In order to complete this template, fill in the blank fields between
% the curly braces or replace already filled in fields with the
% requested information.
%
% e.g. in order to add a title, replace
% \title{} with \title{Evidence of Solitons in Tedium Diboride}
%
% The \author and \address commands can take an optional label
% in square brackets of the form :
%
% \command[label]{}
%
% The text of the abstract should be inserted between the two commands
% \begin{abstract} and \end{abstract}.
%
% Please leave all commands in place even if you don't fill them in.


%----------------------------------------------------------------------
% Do not alter the following two lines

\documentclass[11pt, a4paper]{article}
\usepackage{techrep_ivc}

 
%----------------------------------------------------------------------
% Include any additional packages here!
% package 'graphicx' is automatically included depending on the
%   used compiler (latex, pdflatex), don't include it!!!

% --- added by thesis scaffolding ---
\makeatletter
\@ifundefined{todo}{}{\let\ivc@todo\todo \let\todo\relax}
\makeatother
\usepackage{todonotes}
\newcommand{\ivctodo}[2]{\todo[color=#1!30,inline]{#2}}
% --- end added ---

\begin{document}
%----------------------------------------------------------------------
% Add the following information, which will be shown on the cover sheet

\ivctodo{blue}{thesis outline}
\ivctodo{green}{figures/tables}
\ivctodo{purple}{thesis content}

% Uncomment matching document type
% \type{Technical Report}
% \type{Seminar/Project Computer Vision}
% \type{Seminar/Project Computer Graphics}
% \type{Master Project}
\type{Bachelor Thesis}
%\type{Seminar Paper}

% Number of the technical report (if type == Technical Report)
\reportnr{xxx}

% Name of advisor (if type != Technical Report)
\advisor{Name of Advisor}

% Title
\title{This is the Title}

% Subtitle
\subtitle{This is the subtitle}

% City where the report was created
\repcity{Graz}

% Date of creation
\repdate{\today}

% Keywords that appear below the abstract (if required)
\keywords{Report, Technical report, template, IVC}

% List of authors: List each author using a separate \author{} command. If there 
% is more than one author address, add a label to each author of the form 
% \author[label]{name}. This label should be identical to the corresponding 
% label provided with the \address command. It is not possible to link an author 
% to more than one address!
%
\author[IVC]{Your Name}
%\author[IVC]{Your Name 2}

% List of addresses:  If there is more than one address, list each using a 
% separate \address command using a label to link it to the respective author
% as described above.
%
\newcommand{\TUGn}{Graz University of Technology}
\address[IVC]{Institute of Visual Computing \\ \TUGn, Austria}


% Contact author: If \contact is not defined (uncommented) or empty, the contact  
% information on the title page is suppressed.
%
\contact{Your Name}
\contactemail{youremail@tugraz.at}

%----------------------------------------------------------------------
% Abstract:
\begin{abstract}
\ivctodo{purple}{Abstract plan:\\
- Problem \& gap: Reliable dermoscopic classification is challenging due to severe class imbalance and the presence of out-of-distribution (OOD) inputs that can silently break models.\\
- Approach: Study a multi-stage pipeline (MLP1: skin vs non-skin -> MLP2: 8-class lesion type -> MLP3: benign vs malignant) and a direct final multi-class variant that predicts combined lesion x malignancy labels from the same frozen SAM features via MLP heads; OOD scoring is evaluated with MSP and ODIN.\\
- Data: In-distribution: ISIC 2018--2020 with unified metadata and train/val/test splits. External OOD: Places365 / DTD (textures) (far-OOD) to probe robustness.\\
- Evaluation: Headwise metrics (accuracy, weighted-F1), task-aware pipeline metrics (end-to-end accuracy/F1), confusion matrices; OOD metrics AUROC/AUPR for MSP/ODIN.\\
- Results (Exp1): Skin: >97\% F1 (stable). Lesion: $\sim$51--64\% accuracy, weighted-F1 $\approx$ 0.48--0.60 (macro-F1 much lower due to rare classes). Benign/Malignant: $\sim$77--79\% accuracy, F1 $\sim$0.75--0.79. Task-aware pipeline: $\sim$30\% overall accuracy/F1 due to error propagation. OOD: MSP/ODIN separate ISIC vs Places strongly (AUROC $\approx$ 0.98); ODIN-in-training yields no consistent gain.\\
- Takeaways: The direct final multi-class model matches the multi-stage pipeline while avoiding cascading errors; robustness benefits mainly come from simple MSP/ODIN scoring in far-OOD, not from training-time ODIN. Remaining gaps are driven by minority classes and near-OOD; background-removed inputs and harder OOD benchmarks are promising next steps.\\
- Keywords: Skin lesion classification; multi-task learning; out-of-distribution detection; SAM features; ISIC.}
\ivctodo{blue}{Abstract: One paragraph summary of\\
- Problem: difficulty of reliable skin lesion classification, risk of OOD.\\
- Approach: multi-stage classifier pipeline (skin detection -> lesion classification -> malignancy).\\
- Methods: SAM2 encoder features, MLP heads, dataset curation (ISIC, DermNet, DTD).\\
- Results: validation/test metrics, OOD performance, main findings.\\
- Contribution: task-aware design, background removal, OOD experiments.}
\ivctodo{green}{Abstract: No figures/tables. Text-only summary.}

\end{abstract}
%----------------------------------------------------------------------


%----------------------------------------------------------------------
% Main document 

% example figure
\begin{figure*}[t]
    \centering
    \includegraphics[width=\textwidth]{figures/figure_template.png}
    \caption{Overview. (a) Always provide a good caption in short, comprehensive sentences. The reader should understand your paper just by looking at the figures and the captions.}
\label{fig:overview}
\end{figure*}

\section{Introduction}
\ivctodo{blue}{Introduction: Motivation (medical relevance of skin cancer detection).\\
Problem statement: class imbalance, OOD, visual similarity.\\
Research questions:\\
\quad\,\textbullet\; Can multi-stage classification improve reliability?\\
\quad\,\textbullet\; How does task-aware evaluation compare to independent heads?\\
\quad\,\textbullet\; What is the effect of background removal?\\
\quad\,\textbullet\; How robust are models to OOD?\\
Contributions: bullet list.\\
Thesis structure: one sentence per chapter.}
\ivctodo{green}{Introduction figures:\\
- Figure 1: Conceptual diagram of multi-stage pipeline (skin -> lesion -> bm).\\
- Figure 2: Example images showing skin vs. not-skin (illustrative).}
\ivctodo{purple}{Introduction — bullet outline (ready-to-expand):\\
\textbf{Clinical context \& motivation}\\
- Skin cancer is common and impactful; early, reliable triage from dermoscopy images is critical.\\
- Automated support is attractive but must be reliable under real-world variability (devices, lighting, anatomy).\\
\textbf{Problem framing}\\
- Lesion diagnosis is inherently hierarchical: (1) plausibility check (skin vs. not-skin), (2) lesion type recognition (8 classes), (3) benign vs. malignant risk.\\
- Most ML systems treat this as a single flat task, ignoring hierarchy and error propagation.\\
\textbf{Core challenges}\\
- Severe class imbalance (few SCC/DF/VASC vs. many NV/MEL); majority classes dominate metrics.\\
- Near-OOD \& label noise: non-skin or off-modality images mislabeled as skin; subtle domain shifts.\\
- Cascading errors in pipelines: early mistakes (skin/lesion) invalidate downstream decisions (BM).\\
- Calibration \& OOD detection: high confidence on wrong inputs is unsafe.\\
\textbf{Goal of this work}\\
- Design \& evaluate a dermoscopy classifier that respects hierarchical diagnosis and is robust to OOD and imbalance.\\
\textbf{Research questions (RQs)}\\
- RQ1: Does a multi-stage pipeline (skin -> lesion -> BM) improve reliability vs. independent heads?\\
- RQ2: Can a direct final multi-class model match/surpass the pipeline while avoiding error cascades?\\
- RQ3: How do MSP \& ODIN perform for OOD detection here, and does ODIN-in-training help?\\
- RQ4: What is the effect of dataset strategies (background removal, oversampling) on minorities \& pipeline?\\
- RQ5: Which evaluation protocol (headwise vs. task-aware/pipeline) better reflects clinical use?\\
\textbf{Approach (high-level)}\\
- Use frozen SAM features as a backbone; train light MLP heads on top.\\
- Two modelings: (i) multi-stage (MLP1 skin, MLP2 lesion, MLP3 BM) and (ii) final multi-class (combined lesion x BM).\\
- Reliability via OOD scoring (MSP, ODIN) and task-aware evaluation mirroring clinical flow.\\
\textbf{Data \& splits (summary)}\\
- Unified ISIC 2018--2020 dermoscopy with single metadata (train/val/test).\\
- External sets (Places/DTD) for far-OOD stress tests.\\
- Features cached from the frozen backbone for fast, reproducible runs.\\
\textbf{Evaluation plan}\\
- Headwise: accuracy \& weighted-F1 for skin/lesion/BM, plus confusion matrices.\\
- Task-aware pipeline: end-to-end accuracy/F1 on composed labels; pipeline confusion matrix.\\
- OOD: AUROC/AUPR for MSP/ODIN; qualitative failure modes.\\
\textbf{Contributions}\\
- Unified, reproducible pipeline for hierarchical dermoscopy on SAM features.\\
- Task-aware evaluation exposing error propagation vs. headwise scores.\\
- Direct final multi-class simplifying inference while matching pipeline performance.\\
- Comprehensive OOD study (MSP/ODIN), incl. ODIN-in-training vs. test-time-only.\\
- Ablations: oversampling, mixup, background-removed inputs; thorough confusion diagnostics.\\
\textbf{Scope \& limitations}\\
- Dermoscopy still images; frozen features (no end-to-end finetuning).\\
- Far-OOD separation is easier; near-OOD and minority lesions remain challenging.\\
- Clinical deployment issues (calibration, thresholds, human-in-the-loop) discussed but not fully solved.\\
\textbf{Chapter roadmap}\\
- \S2 Related Work: lesion classification, multi-task/hierarchical ML, OOD.\\
- \S3 Method: data, features, architectures (pipeline vs final multi-class), training \& metrics.\\
- \S4 Experiments: baselines, ablations, OOD protocols.\\
- \S5 Results: quantitative tables, confusion matrices, ROC curves.\\
- \S6 Discussion: insights, limitations, failure modes, implications.\\
- \S7 Conclusion: answers to RQs, key takeaways, future work.}

\section{Related Work}
\ivctodo{blue}{Related Work: medical imaging and skin lesion classification (CNN, transformers).\\
SAM2 and segmentation-based preprocessing.\\
Multi-task and hierarchical classification in ML.\\
OOD detection approaches (MSP, ODIN).\\
Data augmentation and balancing in medical ML.\\
Positioning: where this thesis fits.}
\ivctodo{green}{Related Work tables/figures:\\
- Table 1: Comparison of existing approaches (method, dataset, accuracy, OOD).\\
- Optional figure: Positioning schematic (Venn: Segmentation, Multi-task, OOD).}

\section{Method}
\ivctodo{blue}{Datasets: ISIC 2018/2019/2020, DermNet, SD-198, DTD, ImageNet subset. Unified metadata (mapping diagnosis -> unified\_diagnosis). Preprocessing: cleaning, resizing, background removal. Train/val/test splits (70/15/15).\\
Feature extraction: SAM2 encoder as frozen feature extractor; feature dimensions; caching to .pkl.\\
Model architectures: parallel multi-head model (skin/lesion/bm); task-aware pipeline evaluation logic; final multi-class ablation (11-class).\\
Training setup: losses (LMF, CE), task weights; oversampling vs natural; mixup and augmentations; optimizer, LR scheduler, epochs.\\
Evaluation setup: headwise metrics (accuracy, weighted F1); task-aware metrics (pipeline accuracy/F1); confusion matrices; OOD detection (MSP, ODIN).}
\ivctodo{green}{Method figures/tables:\\
- Table 2: Dataset overview (images, classes, split sizes).\\
- Figure 3: Sample images per dataset (grid with labels).\\
- Figure 4: Class distribution histograms (lesion, benign/malignant).\\
- Figure 5: Image size/aspect ratio distribution.\\
- Figure 6: SAM2 encoder -> feature vector diagram.\\
- Figure 7: Background removal example (before/after).\\
- Figure 8: Parallel multi-head model block diagram.\\
- Figure 9: Task-aware pipeline flowchart (skin -> lesion -> bm).\\
- Figure 10: Final multi-class model (11-class head).\\
- Table 3: Hyperparameters (optimizer, LR, batch, epochs, loss weights).\\
- Figure 11 (optional): Training curriculum schedule.\\
- Figure 12: Headwise vs. task-aware evaluation scheme.\\
- Table 4: Metrics overview (accuracy, weighted F1, AUROC, AUPR).}

\section{Experiments}
\ivctodo{blue}{Baseline: parallel model with weighted F1.\\
Ablations: background removed vs original; final multi-class vs task-aware pipeline; oversampling vs no oversampling; mixup vs no mixup.\\
OOD experiments: MSP and ODIN; performance on DTD / ImageNet.\\
Task-aware evaluation: compare with headwise metrics; pipeline confusion matrices.}
\ivctodo{green}{Experiments figures/tables:\\
- Table 5: Baseline results (skin/lesion/bm acc \& F1).\
- Table 6: Ablation results (bg removal, multi-class, oversampling, mixup).\\
- Figure 13: Bar chart of lesion F1 across ablations.\\
- Table 7: OOD AUROC/AUPR for MSP and ODIN (skin vs. not-skin OOD).\\
- Figure 14: ROC curves MSP vs. ODIN.\\
- Table 8: Headwise vs. task-aware performance.\\
- Figure 15: Pipeline-level confusion matrix.}

\section{Results}
\ivctodo{blue}{Present per-experiment metrics in tables. Confusion matrices (lesion head and pipeline). OOD AUROC/AUPR comparisons.\\
Highlight: best configuration; trade-offs between simplicity (final multi-class) and interpretability (task-aware pipeline); oversampling impact (negative). Link results to research questions.}
\ivctodo{green}{Results figures/tables:\\
- Figure 16: Lesion confusion matrix (parallel multi-head).\\
- Figure 17: Lesion confusion matrix (final multi-class).\\
- Figure 18: Pipeline confusion matrix (task-aware).\\
- Figure 19: Training/validation accuracy and loss curves.\\
- Figure 20: OOD AUROC curve.\\
- Table 9: Summary of all models (baseline, ablations, OOD).}

\section{Discussion}
\ivctodo{blue}{Interpretation: why certain methods helped/hurt.\\
Limitations: dataset biases; SAM2 frozen features; OOD labeling reliability.\\
Practical implications for medical usage.\\
Future work: trainable segmentation; larger datasets; clinical deployment with uncertainty estimation.}
\ivctodo{green}{Discussion figures/tables:\\
- Figure 21: Error analysis examples (misclassified images with predicted vs. GT).\\
- Table 10: Common confusion pairs (e.g., NV vs. MEL).}

\section{Conclusion}
\ivctodo{blue}{Recap problem, approach, contributions. Main findings (weighted F1 better, oversampling harmful, task-aware adds interpretability). Final statement on feasibility of multi-stage lesion classification.}
\ivctodo{green}{Conclusion: No figures/tables (optionally restate Figure 1 pipeline).}

\appendix
\section{Appendix (optional)}
\ivctodo{blue}{Additional confusion matrices. Detailed dataset stats. Training logs / configs.}
\ivctodo{green}{Appendix figures/tables:\\
- Table A1: Full class counts per dataset and split.\\
- Figure A1: Additional confusion matrices (per dataset, per head).\\
- Figure A2: Extended ROC curves (all OOD datasets).\\
- Table A2: Full hyperparameter search log.}





%----------------------------------------------------------------------
% Bibliography/References
\bibliographystyle{plain}
\bibliography{references}

\end{document}
%----------------------------------------------------------------------

