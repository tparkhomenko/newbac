\documentclass[aspectratio=169]{beamer}
%% Choose aspect ratio and other standard options:
% [aspectratio=169] % 16:9 (default)
% [aspectratio=43]  % 4:3 

\usetheme[minimal]{tugraz2018}
%% Choose main theme variant:
% [standard]        % standard (default)
% [institute]       % with institute's graphical acronym on the left
% [minimal]         % with reduced visuals

%% Choose your font style:
%                   % Helvetica (default for Corporate Design)
% [webfont]         % Source Sans Pro (as used on tugraz.at)
% [nofont]          % no font loaded - Computer Modern Sans

%% For more options, see README.pdf

\usepackage[utf8]{inputenc}
\usepackage[english]{babel}
\usepackage{todonotes}  % Add todonotes package
%% Choose your main language:
% [ngerman]   % German
% [english]   % English


%% Add your own packages, macros, etc.
% ...


%% Enter presentation metadata
\title[Skin Image Classification]{Skin Image Classification Pipeline:\\Using SAM2 Encoder Features}
\author{Taisiya Parkhomenko (01650051)}
\date{Bachelor Thesis}
\institute{Institute of Visual Computing (IVC)}
\instituteurl{www.ivc.tugraz.at}

%% Logos
\institutelogo{beamerthemetugraz/institute/kurz}  % IVC logo
% \additionallogo{figures/logo}  % additional logo if needed
% \logobar{Supported by: ...}  % sponsors (titlepage; optional)


\begin{document}

\begin{frame}[plain]
  \maketitle
\end{frame}


\begin{frame}{Outline}
  \tableofcontents
\end{frame}


\section{Introduction}

\begin{frame}{Introduction}
  \begin{itemize}
    \item Two-stage classification pipeline using frozen SAM2 encoder features:
      \begin{itemize}
        \item Stage 1: Skin vs Not Skin Classification
        \item Stage 2: Lesion Group and Malignancy Classification
      \end{itemize}
    \item Large-scale dataset combination:
      \begin{itemize}
        \item ISIC 2019: 157,610 skin lesion images (augmented from 33,569 original)
        \item DTD: 28,200 texture images (augmented from 5,640 original)
      \end{itemize}
    \item Focus on feature extraction using SAM2's powerful vision encoder
  \end{itemize}
\end{frame}

\begin{frame}{Objectives}
  \begin{itemize}
    \item Implement a skin lesion detection pipeline using foundation models
    \item Explore the applicability of SAM2 for medical imaging
    \item Reduce labeling costs by automating classification steps
    \item Combine datasets from different domains for improved robustness
    \item Evaluate transfer learning capabilities for skin lesion detection
    \item Design a modular, hierarchical pipeline for multi-stage classification
  \end{itemize}
\end{frame}


\section{Methodology}

\begin{frame}{Pipeline Overview}
  \begin{itemize}
    \item Two-stage classification using SAM2 encoder features:
      \begin{itemize}
        \item Stage 1: Skin vs Not-Skin (MLP1)
        \item Stage 2: Lesion Group (MLP2) and Malignancy (MLP3)
      \end{itemize}
    \item Feature extraction:
      \begin{itemize}
        \item SAM2 ViT encoder (frozen)
        \item 256-dimensional embeddings
      \end{itemize}
    \item Linear probing to leverage pre-trained features
    \item Segmentation masks for visualization
  \end{itemize}
\end{frame}

\begin{frame}{Datasets}
  \begin{itemize}
    \item ISIC 2019 Dataset:
      \begin{itemize}
        \item 157,610 augmented images (33,569 original)
        \item 5 lesion groups:
          \begin{itemize}
            \item melanocytic: 106,095 images (21,219 original)
            \item non-melanocytic carcinoma: 25,455 images (5,091 original)
            \item keratosis: 22,625 images (4,525 original)
            \item vascular: 1,785 images (357 original)
            \item fibrous: 1,650 images (330 original)
          \end{itemize}
        \item Malignancy labels: benign (96,705), malignant (60,905)
      \end{itemize}
    \item DTD (Describable Textures Dataset):
      \begin{itemize}
        \item 28,200 augmented images (5,640 original)
        \item 47 texture categories
        \item Used as non-skin examples
      \end{itemize}
  \end{itemize}
\end{frame}

\begin{frame}{Data Processing}
  \begin{itemize}
    \item Unified preprocessing pipeline:
      \begin{itemize}
        \item Resize all images to 1024x1024 (SAM format)
        \item 5x augmentation per image:
          \begin{itemize}
            \item Original
            \item Horizontal flip
            \item Rotations (90°, 180°, 270°)
          \end{itemize}
      \end{itemize}
    \item Dataset splits (total 196,045 images):
      \begin{itemize}
        \item Train: 137,122 images (117,490 skin / 19,632 non-skin)
        \item Validation: 29,399 images (25,175 skin / 4,224 non-skin)
        \item Test: 29,524 images (25,180 skin / 4,344 non-skin)
      \end{itemize}
  \end{itemize}
\end{frame}

\begin{frame}{Feature Extraction and Linear Probing}
  \begin{itemize}
    \item SAM2 Feature Extractor:
      \begin{itemize}
        \item Frozen ViT encoder
        \item 256-dimensional embeddings
        \item GPU-accelerated batch processing
      \end{itemize}
    \item Linear Probing Architecture:
      \begin{itemize}
        \item Input: 256-dim SAM features
        \item Hidden layers: [512, 256] with ReLU \& Dropout(0.3)
        \item Output: task-specific classes
      \end{itemize}
  \end{itemize}
\end{frame}


\section{Results}

\begin{frame}{Segmentation Results}
  \begin{itemize}
    \item Qualitative mask visualization
    \item SAM2 performance on different skin conditions
    \item Mask selection effectiveness
  \end{itemize}
  \todo[inline]{Add side-by-side comparison of original images and masks}
\end{frame}

\begin{frame}{Classification Performance}
  \begin{itemize}
    \item Skin vs Not-Skin (MLP1): 99.38\% validation accuracy
      \begin{itemize}
        \item Balanced dataset: 2,000 samples per class
        \item FocalLoss with class weights [1.0, 5.93]
      \end{itemize}
    \item Lesion Type (MLP2): 65.8\% validation accuracy
      \begin{itemize}
        \item Max 2,000 samples per majority class
        \item F1 scores: 0.59-0.72 across 5 classes
        \item FocalLoss with class weights [1.5, 1.5, 1.5, 1.2, 1.2]
      \end{itemize}
    \item Benign/Malignant (MLP3): 72.55\% validation accuracy
      \begin{itemize}
        \item Balanced dataset: 2,000 samples per class
        \item FocalLoss with gamma=2.0
      \end{itemize}
  \end{itemize}
\end{frame}

\begin{frame}{Ablation Study Results}
  \begin{itemize}
    \item Class imbalance handling (MLP3):
      \begin{itemize}
        \item Balanced sampling vs. original distribution
        \item Balanced (4k samples): 72.55\% accuracy, F1 score 0.7255
        \item Original (22k samples): 71.29\% accuracy, F1 score 0.7045
      \end{itemize}
    \item Training time comparison:
      \begin{itemize}
        \item Balanced model: ~1.5 hours
        \item Original distribution: ~4.5 hours (3x slower)
      \end{itemize}
    \item Augmented dataset model still in progress:
      \begin{itemize}
        \item 110,470 images (67,990 benign / 42,480 malignant)
      \end{itemize}
  \end{itemize}
\end{frame}

\begin{frame}{Connection to Objectives}
  \begin{itemize}
    \item Foundation model applicability:
      \begin{itemize}
        \item SAM2 features effective for medical imaging tasks
        \item Linear probing demonstrates strong transfer learning capabilities
      \end{itemize}
    \item Multi-domain dataset integration:
      \begin{itemize}
        \item Successfully combined DTD textures with ISIC skin lesions
        \item Hierarchical pipeline provides modular classification
      \end{itemize}
    \item Automated labeling steps:
      \begin{itemize}
        \item 99.38\% accurate skin detection enables efficient filtering
        \item Modular approach allows targeted fine-tuning
      \end{itemize}
  \end{itemize}
\end{frame}


\section{Conclusion}

\begin{frame}{Key Findings \& Future Work}
  \begin{itemize}
    \item Key Findings:
      \begin{itemize}
        \item SAM2 encoder features highly effective for skin lesion classification
        \item Linear probing with balanced sampling outperforms larger imbalanced datasets
        \item Multi-stage pipeline effective for hierarchical classification
      \end{itemize}
    \item Future Work:
      \begin{itemize}
        \item Investigate prompt point generation using grayscale extrema
        \item Analyze SAM mask filtering heuristics
        \item Explore ensemble methods for improved accuracy
      \end{itemize}
  \end{itemize}
\end{frame}

\begin{frame}{Thank You}
  \centering
  \Large Thank you for your attention!
  
  \vspace{1cm}
  \normalsize
  \begin{itemize}
    \item Supervisor: Dr. Marc Masana Castrillo
    \item Institute of Visual Computing (IVC)
    \item Technical University of Graz
  \end{itemize}
\end{frame}

\end{document}
